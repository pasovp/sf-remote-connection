%!TEX root = ../tesi.tex

\chapter{Il Progetto SF-Remote-Connection}
\label{ch:sfremoteconnection}
In questo capitolo viene descritto il progetto sf-remote-connection, i moduli che lo compongono, le funzionalità offerte e i package java prodotti.

% TODO: cambiare titolo?
\subsection{Descrizione del progetto} 
\label{sub:descrizione}
Il progetto SF-Remote-Connection è una libreria di classi java ospitate, al momento della scrittura di questo documento, sul portale % TODO: aggiungere stile per link
code.google.com per la condivisione di codice open source.
La classi che compongono il progetto sono suddivise in una serie di package. % TODO: decidere se la prossima frase va bene
Alcuni di questi sono pensati per rappresentare una possibile estensione a quelli forniti dal framework stesso e ne riproducono la struttura e le convenzioni sui nomi, gli altri invece affiancano o utilizzano il framework nella costruzione dell'applicazione.
Questi package possono perciò essere raggruppati in una serie di macro-moduli suddivisi per funzionalità e finalità:
% TODO: decidere se rinominare i moduli
\begin{enumerate}
	\item \textbf{Base Communication}
	\item \textbf{RemoteDataCenter Tool}
	\item \textbf{Client}
	\item \textbf{Test Client}
	\item \textbf{Server}         
	\item \textbf{Test Server}
\end{enumerate}

\subsubsection{Base Communication}
\label{subsub:basecommodule}
Questo modulo riunisce le classi che consentono la creazione e la gestione di connessioni TCP/IP tra applicazioni client/server. Ne fa parte anche la classe di utilità \texttt{GenericCommunicator} che oltre a consentire la gestione della connessione assegnatagli la utilizza per fornire funzionalità di lettura e scrittura di messaggi testuali attraverso il canale aperto.
Il modulo è composto dal package \texttt{sfrc.base.communication} ed è una libreria totalmente indipendente dal framework.

\subsubsection{RemoteDataCenter Tool}
\label{subsub:remotedatacentertoolmodule}
Questo modulo raggruppa una serie di classi pensate per essere una estensione del framework.

Forniscono astrazione su i meccanismi di sostituzione dei dataset e sulle richieste


\subsubsection{Client}
\label{subsub:clientmodule}
\subsubsection{Test Client}
\label{subsub:tclientmodule}
\subsubsection{Server}
\label{subsub:servermodule}
\subsubsection{Test Server}
\label{subsub:tservermodule}