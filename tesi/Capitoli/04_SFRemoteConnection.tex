%!TEX root = ../tesi.tex

\chapter{Il Progetto SF-Remote-Connection}
\label{ch:sfremoteconnection}
In questo capitolo viene descritto il progetto sf-remote-connection, i moduli che lo compongono, le funzionalità offerte e i package java prodotti.

% TODO: trovare un posto per queste info
Il progetto SF-Remote-Connection è una libreria di classi java ospitate, al momento della scrittura di questo documento, sul portale % TODO: aggiungere stile per link
code.google.com per la condivisione di codice open source.

% TODO: cambiare titolo?
\section{Moduli} 
\label{sec:moduli}
La classi che compongono il progetto sono suddivise in una serie di package. % TODO: decidere se la prossima frase va bene
Alcuni di questi sono pensati per rappresentare una possibile estensione a quelli forniti dal framework stesso e ne riproducono la struttura e le convenzioni sui nomi, gli altri invece affiancano o utilizzano il framework nella costruzione dell'applicazione.
Questi package possono perciò essere raggruppati in una serie di macro-moduli suddivisi per funzionalità e finalità:
% TODO: decidere se rinominare i moduli
\begin{enumerate}
	\item \textbf{Base Communication}
	\item \textbf{RemoteDataCenter Tool}
	\item \textbf{Client}
	\item \textbf{Test Client}
	\item \textbf{Server}         
	\item \textbf{Test Server}
\end{enumerate}

% TODO: 
%	inserire un'immagine di come i moduli funzionino su vari livelli
%	verificare i nomi dei package

\subsection{Base Communication}
\label{sub:basecommodule}
Questo modulo riunisce le classi che consentono la creazione e la gestione di connessioni TCP/IP tra applicazioni client/server. Ne fa parte anche la classe di utilità \texttt{GenericCommunicator} che oltre a consentire la gestione della connessione assegnatagli la utilizza per fornire funzionalità di lettura e scrittura di messaggi testuali attraverso il canale aperto.
Il modulo è composto dal package \texttt{sfrc.base.communication} ed è una libreria totalmente indipendente dal framework.

\subsection{RemoteDataCenter Tool}
\label{sub:remotedatacentertoolmodule}
Questo modulo raggruppa una serie di classi pensate per essere una estensione del framework e per essere utilizzate principalmente all'interno di una applicazione client.
La sua funzione principale consiste nel fornire un ponte tra l'astrazione del reperimento dati fornita dal framework e il meccanismo di effettivo reperimento dei dati.

La classe chiave del modulo è \texttt{SFRemoteDataCenter}: le richieste di Dataset effettuate al DataCenter vengono passate a questa classe che le esamina verificando che il dato richiesto sia presente nella libreria dell'applicazione. Se il Dataset non è presente, al richiedente è restituito un Dataset sostitutivo temporaneo scelto opportunamente, contemporaneamente viene generata una richiesta e aggiunta ad un buffer di richieste, questo può essere utilizzato da un modulo esterno in grado di effettuare l'effettivo reperimento dei dati.
% TODO: decidere se i 2 package aggiuntivi andrebbero inseriti nel modulo
Il modulo è composto dai package \texttt{shadow.system.data.remote.wip}, \texttt{shadow.system.data.object.wip} e \texttt{shadow.renderer.viewer.wip}.

\subsection{Client}
\label{sub:clientmodule}
Questo modulo raggruppa tutte quelle componenti generiche che possono essere utilizzate all'interno di una qualsiasi applicazione client e che servono ad implementare l'effettivo reperimento dei dati. 
Esso si pone al di sotto del modulo \textbf{RemoteDataCenter Tool} ed utilizza il modulo \textbf{Base Communication} per la gestione del canale di comunicazione e la sua implementazione è pensata per il multi-threading.
Il package che compone questo modulo è \textbf{sfrc.application.client}.

\subsection{Test Client}
\label{sub:tclientmodule}
In questo modulo sono raccolte le implementazioni test

\subsection{Server}
\label{sub:servermodule}
\subsection{Test Server}
\label{sub:tservermodule}