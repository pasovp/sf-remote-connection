%!TEX root = ../tesi.tex

\chapter{Il Progetto SF-Remote-Connection}
\label{ch:sfremoteconnection}
In questo capitolo viene descritto il progetto sf-remote-connection, i moduli che lo compongono, le funzionalità offerte e i package java prodotti.

\section{Informazioni generali} 
\label{sec:geninfo}
In base agli obbiettivi di progetto esposti al paragrafo \ref{sec:obbiettivo}, ciò che è stato prodotto consiste in una estensione dello % TODO: possibile acronimo SF
Shadow Framework e all'implementazione di un sistema di comunicazione dati sfruttato da questa estensione.
Il progetto SF-Remote-Connection è una libreria di classi java ospitate, al momento della scrittura di questo documento, sul portale % TODO: aggiungere stile per link
code.google.com per la condivisione di codice open source.
Il codice è attualmente rilasciato sotto licenza GNU GPL v3.
Si rimanda all'appendice \ref{a:notesoftware} per le versioni delle librerie utilizzate.

% TODO: cambiare titolo?
\section{Moduli} 
\label{sec:moduli}
La classi che compongono il progetto sono suddivise in una serie di package. % TODO: decidere se la prossima frase va bene
Alcuni di questi sono pensati per rappresentare una possibile estensione a quelli forniti dal framework stesso e ne riproducono la struttura e le convenzioni sui nomi, gli altri invece affiancano o utilizzano il framework nella costruzione dell'applicazione.
Questi package possono essere raggruppati in una serie di macro-moduli suddivisi per funzionalità e finalità:
% TODO: decidere se rinominare i moduli
\begin{enumerate}
	\item \textbf{Base Communication}
	\item \textbf{RemoteDataCenter Tool}
	\item \textbf{Client}
	\item \textbf{Test Client}
	\item \textbf{Server}         
	\item \textbf{Test Server}
\end{enumerate}

% TODO: 
%	inserire un'immagine di come i moduli funzionino su vari livelli
%	verificare i nomi e le appartenenze dei package

\subsection{Base Communication}
\label{sub:basecommodule}
Questo modulo riunisce le classi che consentono la creazione e la gestione di connessioni TCP/IP tra applicazioni client/server. Ne fa parte anche la classe di utilità \texttt{GenericCommunicator} che oltre a consentire la gestione della connessione assegnatagli la utilizza per fornire funzionalità di lettura e scrittura di messaggi testuali attraverso il canale aperto.

Il modulo è composto dai package \texttt{sfrc.base.communication} e \texttt{sfrc.base.communication.sfutil}.

\subsection{RemoteDataCenter Tool}
\label{sub:remotedatacentertoolmodule}
Questo modulo raggruppa una serie di classi pensate per essere una estensione del framework e per essere utilizzate principalmente all'interno di una applicazione client.
La sua funzione principale consiste nel fornire un ponte tra l'astrazione del reperimento dati fornita dal framework e il meccanismo di effettivo reperimento dei dati.

La classe chiave del modulo è \texttt{SFRemoteDataCenter}: le richieste di Dataset effettuate al DataCenter vengono passate a questa classe che le esamina verificando che il dato richiesto sia presente nella libreria dell'applicazione. Se il Dataset non è presente, al richiedente è restituito un Dataset sostitutivo temporaneo scelto opportunamente, contemporaneamente viene generata una richiesta e aggiunta ad un buffer di richieste, questo può essere utilizzato da un modulo esterno in grado di effettuare l'effettivo reperimento dei dati.
Il meccanismo dei Dataset sostitutivi viene descritto esaustivamente nella sezione \ref{sec:dataset_sost}.

% TODO: decidere se i 2 package aggiuntivi andrebbero inseriti nel modulo
Il modulo è composto dai package \texttt{shadow.system.data.remote.wip}, \texttt{shadow.system.data.object.wip} e \texttt{shadow.renderer.viewer.wip}.

\subsection{Client}
\label{sub:clientmodule}
Questo modulo raggruppa tutte quelle componenti generiche che possono essere utilizzate all'interno di una qualsiasi applicazione client e che servono ad implementare l'effettivo reperimento dei dati. 
Esso si pone al di sotto del modulo \textbf{RemoteDataCenter Tool} ed utilizza il modulo \textbf{Base Communication} per la gestione del canale di comunicazione e la sua implementazione è pensata per il multi-threading.

Il package che compone questo modulo è \textbf{sfrc.application.client}.

\subsection{Test Client}
\label{sub:tclientmodule}
In questo modulo sono raccolte le implementazioni delle applicazioni di test per le componenti lato client. Questi test sono stati creati principalmente per riprodurre quelli già presenti nel progetto % TODO: aggiungere riferimento online
SF20LiteTestWorld e usati per mostrare le capacità del framework.

Il package che compone questo modulo è \textbf{sfrc.application.client.test}.

\subsection{Server}
\label{sub:servermodule}
Similmente al modulo per le componenti client, in questo vengono raggruppate delle componenti generiche utili alla realizzazione di una applicazione server. Queste componenti si pongono da tramite tra l'applicazione e il modulo di \textbf{Base Communication} tramite cui realizzano l'effetivo trasferimento dei Dataset verso il client connesso.
Anche in questo caso l'implementazione è pensata per il funzionamento multi-thread in parallelo con l'applicazione principale che può così gestire più client connessi contemporaneamente ed eseguire altre operazioni.
Vengono fornite infine anche delle interfacce utili per effettuare l'inizializzazione dei dati e per configurare il protocollo di comunicazione.

Il modulo è composto dal package \texttt{sfrc.application.server}

\subsection{Test Server}
\label{sub:tservermodule}
Di questo modulo fanno parte le implementazioni di server e applicazioni di test utilizzati per verificare il comportamento delle componenti lato server. Oltre a queste vengono usati per testare i client in differenti condizioni di comunicazione simulate dai server, come ad esempio una risposta a singhiozzo, ecc.

Di questo modulo fanno parte i package \texttt{sfrc.application.server.test} e \texttt{sfrc.application.server.task}.

\section{Dataset sostitutivi}
\label{sec:dataset_sost}
Per evitare che, in seguito alla richiesta di un Dataset non presente nella libreria locale di un'applicazione, i moduli richiedenti rimanessero in attesa dei dati bloccando di fatto l'esecuzione, è stato deciso di realizzare un meccanismo di sostituzione-aggiornamento delle richieste.
Successivamente ad una richiesta il RemoteDataCenter restituisce, attraverso la callback del richiedente, un Dataset sostitutivo temporaneo di tipo compatibile a quello richiesto.
Contemporaneamente viene generata una richiesta remota che attende di essere evasa. Quando i dati effettivi arrivano dalla rete viene eseguito un update del dato richiamando nuovamente la callback del richiedente.
Dato che più moduli dell'applicazione potrebbero fare richiesta dello stesso Dataset, tutte le callback dei richiedenti vengono memorizzate e, al momento dell'update, richiamate in successione.
% TODO: decidere se parlare dell'Updater
Per permettere il funzionamento di questo automatismo si è resa necessaria la realizzazione di un nuovo tipo di Dataset, l'SFDatasetReplacement, e di una libreria di Dataset sostitutivi.

Utilizzato all'interno di una ObjectsLibrary un DatasetReplacement permette di realizzare una lista di sostituzione che associa il nome di un Dataset "Alfa" richiesto, a quello di un Dataset sostitutivo di default "Beta" e ad un timestamp.
L'associazione tra nomi viene usata per una ricerca diretta del dato sostitutivo da utilizzare, mentre il timestamp è stato introdotto per lo sviluppo futuro di logiche di aggiornamento della lista di sostituzione.
Le liste di sostituzione sono state inoltre rese configurabili attraverso file XML leggibili dal decoder interno del framework.
% TODO: parlare del tool di generazione automatico?

La libreria dei Dataset di default è stata invece realizzata progressivamente durante l'implementazione dei test, quando si rendeva necessaria la costruzione di un Dataset specifico dato che quelli già realizzati erano di tipo incompatibile.
% TODO: specificare cosa si intende per tipo incompatibile

L'utilizzo di questo meccanismo richiede necessariamente una fase di inizializzazione in cui viene fatto il download o il caricamento da disco locale della lista di sostituzione e della libreria dei Dataset di default. Avendo la possibilità di estendere il protocollo di comunicazione in modo da rendere espandibili queste due librerie durante l'esecuzione, è possibile mantenere la loro dimensione iniziale contenuta.


\section{I Package java} 
\label{sec:sfrc_packages}
