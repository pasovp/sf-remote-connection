%!TEX root = ../tesi.tex

\chapter{Test e Risultati}
\label{ch:testerisultati}
Dopo aver definito nel precedente capitolo la struttura del progetto, vengono qui presentati i test effettuati ed i risultati ottenuti.


\section{Test Client}
\label{sec:tclientmodule}

\textbf{Test Client}

In questo modulo sono raccolte le implementazioni delle applicazioni di test per le componenti lato client. Questi test sono stati creati principalmente per riprodurre quelli gi\`a presenti nel progetto % TODO: aggiungere riferimento online
SF20LiteTestWorld e usati per mostrare le capacit\`a del framework.
Ne fanno parte anche le classi che implementano i task del protocollo di comunicazione lato client, per la cui trattazione fare riferimento alla sezione \ref{sub:comprotocol}.

Il package che compone questo modulo \`e \texttt{sfrc.application.client.test} e e \texttt{sfrc.application.client.task}.



\section{Test Server}
\label{sec:tservermodule}

\textbf{Test Server}

Di questo modulo fanno parte le implementazioni di server e applicazioni di test utilizzati per verificare il comportamento delle componenti lato server. Oltre a queste vengono usati per testare i client in differenti condizioni di comunicazione simulate dai server, come ad esempio una risposta a singhiozzo, ecc.
Ne fanno parte anche le classi che implementano i task del protocollo di comunicazione lato server, fare riferimento alla sezione \ref{sub:comprotocol} per una trattazione pi\`u approfondita.

Di questo modulo fanno parte i package \texttt{sfrc.application.server.test} e \texttt{sfrc.application.server.task}.



