%!TEX root = ../tesi.tex

% TODO: inserire le versioni dei software
\chapter{Note sul Software}
\label{a:notesoftware}

\section{Codice sorgente del progetto SF-Remote-Connection}
\label{sec:sfrcsource}
Il progetto SF-Remote-Connection \`e una libreria di classi java ospitate, al momento della scrittura di questo documento, sul portale Google Code per la condivisione di codice open source all'indirizzo \url{http://code.google.com/p/sf-remote-connection/}.
Il codice \`e attualmente rilasciato sotto licenza GNU GPL v3.

\section{Versioni dei software utilizzati}
\label{sec:versionisw}

\subsection{Shadow Framework 2.0}
\label{sub:sfsource}
La versione di riferimento del framework \`e disponibile all'indirizzo \url{http://code.google.com/p/shadowframework20lite/}. Il codice \`e attualmente rilasciato sotto licenza GNU GPL v3.

La revisione utilizzata \`e la: r201.

\subsection{JOGL}
\label{sub:jogl}
JOGL \`e un binding Java open source e multipiattaforma per l'\ac{API} grafica OpenGL.
I suoi sorgenti e i binari sono disponibili sul sito \url{http://jogamp.org/jogl/www/}.

La versione utilizzata \`e la build: 2.0-b66-20121101.

\subsection{Sviluppo del progetto}
\label{sub:sviluppo}
Nel corso del progetto sono stati usati questi strumenti software e linguaggi:
\begin{itemize}
	\item \textbf{Eclipse}, IDE versione Indigo Service Release 2 \cite{book:eclipse};
	\item \textbf{SVN}, tool di versioning \cite{book:svn};
	\item \textbf{Git}, tool di versioning \cite{site:git};
	\item \textbf{Java}, linguaggio di programmazione ad oggetti \cite{book:java};
\end{itemize}
Utili alla comprensione degli argomenti trattati sono stati i seguenti testi tecnici \cite{book:openglbible, book:librografica}
