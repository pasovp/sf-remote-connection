%!TEX root = ../tesi.tex

\chapter{Introduzione}
\label{ch:introduzione}
% TODO: decidere se togliere o come cambiare le righe iniziali di ogni capitolo
In questo capitolo vengono presentati brevemente i contenuti dei diversi capitoli della presente tesi di laurea, vengono fornite una descrizione iniziale del progetto di tesi e dei suoi obbietti e presentate alcune considerazioni generali in merito al progetto stesso.

\section{Lo Shadow Framework}
\label{sec:sfintro}
% TODO: acronimi
Questa tesi è stata realizzata nell'ambito dello sviluppo dello Shadow Framework 2.0, progetto ideato e sviluppato dall'ingegner Alessandro Martinelli.
Lo \ac{SF} è un framework per lo sviluppo di applicazioni che fanno uso di grafica tridimensionale sia in ambito scientifico che per l'intrattenimento.
Tra le sue caratteristiche principali vi sono la portabilità, la possibilità di lavorare sia con dati locali che con dati in remoto, la capacità di sfruttare in maniera estesa le caratteristiche delle moderne \ac{GPU}.\footnote{Una \ac{GPU} è un microprocessore dedicato alla generazione delle immagini visualizzate sullo schermo di un dispositivo, alleggerendo da questo carico il processore principale.}
% TODO: decidere se aggiungere la fonte http://www.springerreference.com/docs/html/chapterdbid/305953.html

Una descrizione più dettagliata delle caratteristiche del framework viene fornita nel capitolo \ref{ch:shadowframework}.

%
% TODO: DA RIVEDERE, SE È IL CASO DI RISCRIVERE IL DISCORSO IN MANIERA PIÙ OGGETTIVA
%
\section{Obbiettivo del progetto}
\label{sec:obbiettivo}
L'obbiettivo del progetto di tesi nasce dall'idea di produrre un'applicazione dimostrativa delle funzionalità di rete offerte dallo Shadow Framework.
L'applicazione che si voleva ottenere era una coppia client-server in cui il server fosse in grado di gestire connessioni simultanee da parte di un numero indefinito di client. 
Ogni client, ottenuta una connessione con il server, doveva assere in grado di visualizzare una scena iniziale navigabile, richiedendo solamente i dati relativi all'ambiente in prossimità di un eventuale avatar.
Successivamente si voleva analizzare due possibili approcci: uno in cui, secondo le necessità, il client avrebbe richiesto al server i dati aggiuntivi riguardo la scena, ad esempio una volta raggiunti i bordi dell'ambiente, oppure un secondo in cui il server, comunicando attivamente con il client, tiene traccia degli spostamenti nella navigazione e fosse in grado di comporre in modo dinamico dei pacchetti di dati prevedendo le necessità del client.

Le astrazioni del layer dati del framework sono state progettate specificatamente per consentire lo sfruttamento della comunicazione di rete, ma fino a quel momento non era stata fatta alcuna specifica implementazione che la utilizzasse. Si desiderava perciò produrre questo tipo di applicazione anche per individuare e correggere i probabili bug presenti nel codice dovuti a vincoli di sincronizzazione non evidenziati dai test effettuati con dati sulla macchina locale.

L'obbiettivo della tesi è così diventato quello di produrre dei moduli di libreria che fossero utili allo sviluppo di applicazioni client-server.

La progettazione di questi moduli si è focalizzata su alcuni punti cardine che rappresentano la chiave dell'aspetto di sviluppo legato al progetto. Dato che la struttura del framework è ideata con l'obbiettivo di essere fortemente estendibile, si voleva che i moduli utilizzassero i meccanismi e le astrazioni previste.
Si desiderava inoltre che i moduli fossero a loro volta progettati per l'estendibilità e il riutilizzo del codice.


\section{Considerazioni generali}
\label{sec:considerazioni}
% questa parte probabilmente va accorpata alla precedente
MENZIONARE I TEST

Qui si può fare un discorso sulle parti del framework su cui è necessario focalizzarsi facendo riferimenti ai capitoli 2 e 3.

\section{Organizzazione del documento}
\label{sec:orgtesi}
Il presente documento è organizzato secondo la seguente suddivisione in capitoli:
\begin{itemize}
	\item  \textbf{Capitolo \ref{ch:shadowframework}:} in cui viene presentata una panoramica generale sullo Shadow Framework 2.0 in rapporto al panorama generale sui framework di programmazione di grafica tridimensionale real-time.
	\item  \textbf{Capitolo \ref{ch:gestionedati}:} in descritta l'astrazione di gestione dei dati interna al framework e come essa viene utilizzata dalle applicazioni \ac{SF}.
	\item  \textbf{Capitolo \ref{ch:sfremoteconnection}:} in cui viene descritto il progetto Sf-Remote-Connection, i moduli che lo compongono, le funzionalità offerte e i package java prodotti.
	\item  \textbf{Capitolo \ref{ch:testerisultati}:}
	\item  \textbf{Capitolo \ref{ch:conclusioni}:}
\end{itemize}


