%!TEX root = ../tesi.tex

\chapter{Introduzione}
\label{ch:introduzione}
% TODO: decidere se togliere o come cambiare le righe iniziali di ogni capitolo
In questo capitolo vengono presentati brevemente i contenuti dei diversi capitoli della presente tesi di laurea, vengono fornite una descrizione iniziale del progetto di tesi e dei suoi obbietti e presentate alcune considerazioni generali in merito al progetto stesso.

\section{Lo Shadow Framework}
\label{sec:sfintro}
% TODO: acronimi
Questa tesi è stata realizzata nell'ambito dello sviluppo dello Shadow Framework 2.0, progetto ideato e sviluppato dall'ingegner Alessandro Martinelli.
Lo \ac{SF} è un framework per lo sviluppo di applicazioni che fanno uso di grafica tridimensionale sia in ambito scientifico che per l'intrattenimento.
Tra le sue caratteristiche principali vi sono la portabilità, la possibilità di lavorare sia con dati locali che con dati in remoto, la capacità di sfruttare in maniera estesa le caratteristiche delle moderne \ac{GPU}.\footnote{Una \ac{GPU} è un microprocessore dedicato alla generazione delle immagini visualizzate sullo schermo di un dispositivo, alleggerendo da questo carico il processore principale.}
% TODO: decidere se aggiungere la fonte http://www.springerreference.com/docs/html/chapterdbid/305953.html

Una descrizione più dettagliata delle caratteristiche del framework viene fornita nel capitolo \ref{ch:shadowframework}.

%
% TODO: DA RIVEDERE, SE È IL CASO DI RISCRIVERE IL DISCORSO IN MANIERA PIÙ OGGETTIVA
%
\section{Obbiettivo del progetto}
\label{sec:obbiettivo}
L'obbiettivo del progetto di tesi nasce dall'idea di produrre un'applicazione dimostrativa delle funzionalità di rete offerte dallo Shadow Framework.
L'applicazione che si voleva ottenere era una coppia client-server in cui il server fosse in grado di gestire connessioni simultanee da parte di un numero indefinito di client. 
Ogni client, ottenuta una connessione con il server, doveva assere in grado di visualizzare una scena iniziale navigabile, richiedendo solamente i dati relativi all'ambiente in prossimità di un eventuale avatar.
Successivamente si voleva analizzare due possibili approcci: uno in cui, secondo le necessità, il client avrebbe richiesto al server i dati aggiuntivi riguardo la scena, ad esempio una volta raggiunti i bordi dell'ambiente, oppure un secondo in cui il server, comunicando attivamente con il client, tiene traccia degli spostamenti nella navigazione e fosse in grado di comporre in modo dinamico dei pacchetti di dati prevedendo le necessità del client.

Le astrazioni del layer dati del framework sono state progettate specificatamente per consentire lo sfruttamento della comunicazione di rete, ma fino a quel momento non era stata fatta alcuna specifica implementazione che la utilizzasse. Si desiderava perciò produrre questo tipo di applicazione anche per individuare e correggere i probabili bug presenti nel codice e dovuti a vincoli di sincronizzazione ancora non affrontati dato che fino a quel momento tutti i test erano stati effettuati con dati sulla macchina locale.

L'obbiettivo della tesi è così diventato quello di produrre appunto un'implementazione delle astrazioni del framework che utilizzasse la comunicazione di rete per reperire i dati da visualizzare.


\section{Considerazioni generali}
\label{sec:considerazioni}
% questa parte probabilmente va accorpata alla precedente


\section{Organizzazione della tesi}
\label{sec:orgtesi}