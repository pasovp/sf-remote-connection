%!TEX root = ../tesi.tex

\chapter{Conclusioni}
\label{ch:conclusioni}
L'obbiettivo del progetto di tesi \`e stato quello di progettare e produrre una serie di moduli software di supporto per lo sviluppo di applicazioni di grafica tridimensionale orientate al Web, che fanno uso dello Shadow Framework. I moduli prodotti si sono dimostrati efficaci allo scopo, consentendo di riprodurre i test effettuati con dati grafici in locale anche nella condizione in cui i dati sono memorizzati su di un server remoto. 
Per consentire la parallelizzazione delle richieste sono stati affrontati numerosi problemi di sincronizzazione per cui \`e stato di fondamentale importanza acquisire familiarit\`a la programmazione multi-thread in Java.


Il lavoro di progettazione e sviluppo del codice \`e stato guidato dai principi di riutilizzo e modularit\`a, sfruttando pratiche quali i \textit{Design Pattern} e le metodologie di sviluppo agile.



\section{Sviluppi futuri}
I possibili sviluppi futuri del lavoro compiuto durante questa tesi sono molteplici e orientati in diverse direzioni. Innanzitutto vi \`e la possibilit\`a di espandere i moduli \textbf{Client} e \textbf{Server} in modo che al loro interno vi siano degli strumenti per gestire in modo pi\`u completo le sessioni di comunicazione. Con questo si intende la possibilit\`a di instaurare una comunicazione non legata strettamente alle richieste di dati, ma orientata allo scambio di messaggi di controllo, in modo di consentire l'interazione di utenti differenti che condividono la stessa simulazione.

Un'altra possibilit\`a di sviluppo consiste nell'integrare nel modulo \textbf{Base Communication} lo sfruttamento di protocolli di comunicazione pi\`u complessi e potenti, come ad esempio il protocollo peer-to-peer sfruttato nella tesi \cite{tesi:truzzi}.

La possibilit\`a di editare via xml le liste di sostituzione dei dataset non eliminano il problema che la loro produzione sia lunga e laboriosa, soprattutto quando gli scenari diventano ricchi di modelli tridimensionali. Per questo motivo sarebbe necessario un tool in grado di esaminare i file che descrivono gli scenari e generare in automatico la lista.

\`E in fase di realizzazione un'estensione del protocollo di comunicazione che consenta al server di rispondere alle richieste con una libreria contenente un insieme di Dataset.

Oltre agli sviluppi direttamente collegati al progetto di tesi, i test effettuati hanno evidenziato una serie di elementi del framework che potrebbero avere la necessit\`a di correzioni.

% TODO: eliminare i dettagli ed espandere il generico.