%!TEX root = ../tesi.tex

\chapter{Gestione dei dati nello Shadow Framework 2.0}
\label{ch:gestionedati}

% TODO: ampliare l'introduzione
La gestione dei dati è un compito molto importante all'interno del framework. Attraverso l'utilizzo di un layer di gestione dati astratto, ogni mudulo del framework può essere salvato e caricato da file o trasferito attraverso un qualsiasi flusso di dati.
In questo capitolo viene presentata l'astrazione utilizzata dallo Shadow Framework nella gestione dei dati, le funzionalità messe a disposizione ed i principali package e moduli coinvolti.

\section{L'astrazione della gestione dati}
\label{sec:astrazione}
All'interno di un'applicazione \ac{SF} l'unità base di dati può essere identificata con quello che viene definito SFDataset (\ref{sub:sfdataset}). Con Dataset si identifica quasi ogni tipo di dato, sia grafico che non, utilizzato all'interno del framework.
La gestione dei dataset viene effettuata mediante un meccanismo centralizzato: ogni applicazione in esecuzione possiede un'istanza di SFDataCenter, questa classe è un oggetto \textit{Singleton}\footnote{Con \textit{Singleton} si intende il design pattern omonimo descritto in maniera più accurata nella sezione \ref{sub:singleton}}
a cui ogni componente può accede per richiedere operazioni sui Dataset di interesse, operazioni che possono essere la lettura o la scrittura da uno stream specifico, la richiesta di una particolare istanza di un Dataset, identificata per nome, o la richiesta di una nuova istanza di Dataset, identificata per tipo.
L'oggetto Singleton espone queste funzionalità traducendole internamente con chiamate ad una factory concreta\footnote{Si fa riferimento al pattern di programmazione \textit{Abstract Factory} descritto nella sezione \ref{sub:abstractfactory}}
di Dataset e ad una istanza dell'interfaccia SFIDataCenter, creando un'astrazione su come i Dataset siano effettivamente costruiti e reperiti.
La factory concreta deve essere un'implementazione dell'interfaccia SFAbstractDatasetFactory in grado di istanziare, leggere o scrivere ogni tipo di Dataset utilizzato dall'applicazione.
L'istanza dell'interfaccia SFIDataCenter tiene traccia dei Dataset istanziati con nome, restituendone un riferimento a chi ne fa richiesta attraverso la chiamata a funzioni di callback.

% TODO: aggiungere immagine


\section{Il package \texttt{shadow.system.data}}
\label{sec:shadow_system_data}
Questo package contiene una serie di classi ed interfacce su cui si basa l'astrazione dei dati del framework.

\subsection{SFDataObject}
\label{sub:sfdataobject}
Uno dei moduli principali del package è \texttt{SFDataObject}, che rappresenta un'interfaccia con funzionalità di base comuni ad ogni oggetto che contiene dati. 
Ogni oggetto di questo tipo può perciò:
\begin{itemize}
	\item essere scritto su di un \texttt{SFOutputStream};
	\item essere letto da un \texttt{SFInputStream};
	\item essere clonato;
\end{itemize}
I DataObject si basano sul \textit{Composite Pattern}\ref{sub:composite}: possono essere semplici o contenere un insieme di oggetti figli, il fatto che sia gli oggetti complessi che gli oggetti semplici condividano la stessa interfaccia permette di trattare gli oggetti in modo uniforme. Un oggetto contenitore dovrà semplicemente richiamare lo stesso metodo di interfaccia per tutti gli oggetti figli i quali, se oggetti semplici, hanno la responsabilità di implementare l'algoritmo per leggere o scrivere se stessi da uno stream.

Tutti i componenti SF utilizzano dei DataObject per incapsulare i dati in modo che questi ultimi possano essere letti e scritti utilizzando stream appropriati.

\subsection{SFDataset}
\label{sub:sfdataset}
Un altro modulo importante per la gestione dei dati è \texttt{SFDataset}. Un Dataset è un oggetto che contiene un DataObject e informazioni sul proprio tipo, rappresentato tramite una stringa. 
L'interfaccia SFDataset definisce un'interfaccia per oggetti di questo tipo, la quale consente di accedere al nome del tipo specifico, al DataObject contenuto e di creare una nuova istanza delle stesso tipo.
A loro volta i Dataset possono essere incapsulati in un DataObject usando un oggetto \texttt{SFDatasetObject}.


\subsection{SFAbstractDatasetFactory}
\label{sub:sfabstractdatasetfactory}
Questa interfaccia definisce le operazioni base richieste ad una DatasetFactory, queste operazioni consistono in:
\begin{itemize}
	\item lettura/scrittura di un Dataset da uno stream
	\item la creazione di una nuova istanza di un Dataset specificato per tipo
\end{itemize}

\subsection{SFGenericDatasetFactory}
\label{sub:sfgenericdatasetfactory}
Consiste in una implementazione concreta di default dell'interfaccia \texttt{SFAbstractDatasetFactory}.
Per consentire il riutilizzo del codice è stata resa configurabile: l'aggiunta di un metodo addSFDataset() consente di generare, in un oggetto GenericDatasetFactory, un elenco di Dataset istanziabili.
Quando verranno chiamati i metodi dell'interfaccia SFAbstractDatasetFactory sull' oggetto GenericDatasetFactory diviene sufficiente richiamare il metodo opportuno del Dataset del tipo richiesto o del DataObject in esso contenuto.

\subsection{SFIDataCenter}
\label{sub:sfidatacenter}
L'interfaccia \texttt{SFIDataCenter} fornisce l'astrazione di una Mappa di Dataset identificati attraverso il proprio nome, attraverso di essa possiamo chiedere di recuperare un Dataset ad un oggetto che la implementa.
Quest'oggetto non deve restituire direttamente il Dataset recuperato, ma deve farlo attraverso un meccanismo di callback ad una implementazione dell'interfaccia \texttt{SFDataCenterListener} passata come parametro, nel momento in cui il dato è disponibile.

\subsection{SFDataCenterListener}
\label{sub:sfdatacenterlistener}
Questa interfaccia definisce la callback che un componente deve implementare per effettuare una richiesta al DataCenter.
Questa callback viene richiamata quando il Dataset richiesto è pronto.

\subsection{SFDataCenter}
\label{sub:sfdatacenter} 
% TODO: evitare di ripetere quanto detto in sec:astrazione
Il \textbf{DataCenter} é il nodo fondamentale della gestione dei dati all'interno del framework. 

É un oggetto \textit{singleton} a cui le applicazioni accedono per richiedere i Dataset di cui hanno bisogno fornendo un'astrazione su come i dati sono effettivamente reperiti.

Per poter funzionare, al DataCenter deve essere fornita un'implementazione per:
\begin{itemize}
	\item \texttt{SFAbstractDatasetFactory}
	\item \texttt{SFIDataCenter}
\end{itemize}

Come precedentemente esposto, l'implementazione di \texttt{SFAbstractDatasetFactory} deve essere essere una factory in grado di generare istanze di tutti i tipi di Dataset necessari all'applicazione.

Questo tipo di astrazione permette di separare la logica di utilizzo del Dataset da quella di come esso viene reperito, consentendo ad una applicazione di usare dati locali o dati di rete semplicemente cambiando l'implementazione di \texttt{SFIDataCenter}.

\subsection{SFObjectsLibrary}
\label{sub:sfobjectslibrary}
É usata per memorizzare un set di Dataset ed al suo interno ogni elemento è identificato tramite un nome univoco.
Un \texttt{SFObjectsLibrary} è a sua volta un Dataset, così che un ObjectsLibrary possa essere contenuta in altre ObjectsLibrary.
É possibile, ad esempio, utilizzare una ObjectLibrary all'interno di implementazione di \texttt{SFIDataCenter} per creare una mappa di Dataset necessari al funzionamento di un'applicazione.

\subsection{SFLibraryreference}
\label{sub:sflibraryreference}
Un LibraryReference è un DataObject che può essere usato da qualsiasi componente per avere un riferimento ad un Dataset memorizzato in una libreria.

\subsection{SFInputputStream e SFOutputStream}
\label{sub:sfinoutstream}
Queste interfacce definiscono le operazioni necessarie che uno stream di input o di output deve implementare affinchè sia possibile leggere o scrivere su di esso dei DataObject.
